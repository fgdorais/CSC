\documentclass[csc]{subfiles}
\setcounter{section}{7}
\begin{document}

\section{Compact Spaces}

For compactness of countable second-countable spaces, we would like to use the traditional definition: every open cover has a finite subcover.
However, arbitrary open covers are third-order objects which are inaccessible to second-order arithmetic.
On the one hand, since the spaces we are interested in are countable, it is reasonable to consider countable compactness: every countable open cover has a finite subcover.
On the other hand, since the spaces we are interested in are second-countable, it is also reasonable to only consider covers using basic opens.
Interestingly, these two reasonable approaches diverge in \RCA.

\begin{definition}[\RCA]\label{D:Compact}
  We say that the csc space \(\csc{X}\) is \emph{compact} if for every uniformly enumerable family \(\seq{W_n}_{n \in \N}\) of open \eset{s} such that \(X = \bigcup_{n \in \N} W_n\) there is an \(n_0 \in \N\) such that \(X = \bigcup_{n \leq n_0} W_n.\)
\end{definition}

\begin{proposition}[\RCA]\label{P:CompactImage}
  Let \(\csc{X}\) and \(\csc{Y}\) be csc spaces.
  If there is a continuous surjection \(f:\csc{X}\to\csc{Y}\) and \(\csc{X}\) is compact then \(\csc{Y}\) is also compact.
\end{proposition}

\begin{proposition}[\RCA]\label{P:CompactSubspace}
  Let \(\csc{X}\) be a compact csc space.
  If \(Y\) is a closed subset of \(\csc{X}\) then the subspace \(\csc{Y}\) is also compact.
\end{proposition}

In practice, it often suffices to consider basic open covers rather than more general open covers.
Indeed, if \(\seq{W_n}_{n \in \N}\) is an open cover of the csc space \((X,\mathcal{U},k)\) then \(\seq{U_i}_{i \in A}\) is a basic open refinement where \[A = \set{ i \in I : (\exists n)(U_i \subseteq W_n)}.\]
Unfortunately, the definition of \(A\) is \(\Sigma^0_2\) and therefore \(A\) may fail to be a set or even an \eset{} in subsystems of \ACA.

\begin{definition}[\RCA]
  Let \(\csc{X}\) be a csc space with basis \(\mathcal{U} = \seq{U_i}_{i \in I}.\)
  We say that \((X,\mathcal{U},k)\) is \emph{basically compact} if for every \eset{} \(A \subseteq I\) such that \(X = \bigcup_{i \in A} U_i\) there is a \fset{} \(a \subseteq A\) such that \(X = \bigcup_{i \in a} U_i.\)
\end{definition}

\noindent
A priori, basic compactness depends on the choice of base for the topology.
However, it follows from the next proposition shows that basic compactness is invariant under effective homeomorphisms.

\begin{proposition}[\RCA]\label{P:BCompactImage}
  Let \(\csc{X}\) and \(\csc{Y}\) be csc spaces.
  If there is an effectively continuous surjection \(f:\csc{X}\to\csc{Y}\) and \(\csc{X}\) is basically compact then \(\csc{Y}\) is basically compact too.
\end{proposition}

\begin{proof}
  Let \(\mathcal{U} = \seq{U_i}_{i \in I}\) and \(\mathcal{V} = \seq{V_j}_{j \in J}\) be the respective bases of \(\csc{X}\) and \(\csc{Y},\) and let \(\phi:X \times J \to I\) witness the effective continuity of the surjective map \(f:X \to Y.\)
  Given an \eset{} \(B \subseteq J\) such that \(Y = \bigcup_{j \in B} V_j,\) consider the \eset{} \[A = \set{\phi(x,j) : j \in B \land f(x) \in V_j}.\]
  Note that \(X = \bigcup_{i \in A} U_i.\)
  Since \((X,\mathcal{U},k)\) is basically compact, there is a \fset{} \(a \subseteq A\) such that \(X = \bigcup_{i \in a} U_i.\)
  By definition of \(A,\) there is a \fset{} \(b \subseteq B\) such that \[a \subseteq \set{\phi(x,j) : j \in b \land f(x) \in V_j}.\]
  We claim that \(Y = \bigcup_{j \in b} V_j.\)
  Given \(y_0 \in Y,\) first find \(x_0 \in X\) such that \(f(x_0) = y_0\) and then find \(i \in a\) such that \(x_0 \in U_i.\)
  Note that \(i = \phi(x',j)\) for some \(j \in b\) and some \(x \in X\) such that \(f(x) \in V_j.\)
  By definition of \(\phi,\) we then have \(U_i \subseteq f^{-1}[V_j]\) which means that \(y_0 = f(x_0) \in V_j.\)
\end{proof}

\begin{proposition}[\RCA]\label{P:BCompactSubspace}
  Let \(\csc{X}\) be a basically compact csc space.
  If \(Y\) is an effectively closed subset of \(X\) then the subspace \(\csc{Y}\) is also basically compact.
\end{proposition}

\begin{proof}
  Let \(\mathcal{U} = \seq{U_i}_{i \in I}\) be the basis of \(\csc{X}.\)
  Say, \(A_1 \subseteq I\) is an \eset{} such that \(X \rem Y = \bigcup_{i \in A_1} U_i.\)
  
  Given an \eset{} \(A_0 \subseteq I\) such that \(Y \subseteq \bigcup_{i \in A_0} U_i,\) we have \(X = \bigcup_{i \in A_0 \cup A_1} U_i.\) 
  Since \((X,\mathcal{U},k)\) is compact, there is a \fset{} \(a \subseteq A_0 \cup A_1\) such that \(X = \bigcup_{i \in a} U_i.\)
  Since \(Y \cap U_i = \varnothing\) for every \(i \in A_1,\) we necessarily have \(Y \subseteq \bigcup_{i \in a_0} U_i\) where \(a_0 = a \cap A_0.\)
\end{proof}

\end{document}
