\documentclass[csc]{subfiles}
\setcounter{section}{6}
\begin{document}

\section{Hausdorff Spaces}

A topological space is \emph{Hausdorff} (or \T2) if any two distinct points in the space have disjoint open neighborhoods.
Of course, we may require both of these neighborhoods to be basic, so a csc space \(\csc{X} = (X,\seq{U_i}_{i \in I},k)\) is Hausdorff if for any \(x_0,x_1 \in X,\) if \(x_0 \neq x_1\) then there are indices \(i_0,i_1 \in I\) such that \(x_0 \in U_{i_0},\) \(x_1 \in U_{i_1}\) and \(U_{i_0} \cap U_{i_1} = \varnothing.\)
Since \(U_{i_0} \cap U_{i_1} = \varnothing\) is a \(\Pi^0_1\) statement, there may be no effective way to get two such indices \(i_0,i_1\) from the pair of points \(x_0,x_1.\)


\begin{definition}[\RCA]
  A csc space \(\csc{X}\) with basis \(\mathcal{U}=\seq{U_i}_{i\in I}\) is \emph{effectively Hausdorff} if there are functions \(h_0,h_1:X \times X \to I\) such that if \(x_0 \neq x_1\) then \(x_0 \in U_{h_0(x_0,x_1)},\) \(x_1 \in U_{h_1(x_0,x_1)}\) and \(U_{h_0(x_0,x_1)} \cap U_{h_1(x_0,x_1)} = \varnothing.\)
\end{definition}

\noindent
The following gives an elegant characterization of effectively Hausdorff csc spaces.

\begin{proposition}[\RCA]\label{P:HausdorffDiagonal}
  A weak or strong csc space \(\csc{X}\) is effectively Hausdorff if and only if the diagonal is effectively closed in the product space \(\csc{X}\times\csc{X}.\)
\end{proposition}

\begin{proof}
  Write \(\csc{X} = (X,\seq{U_i}_{i\in I},k).\)
  Suppose that \(A \subseteq I \times I\) is an \eset{} such that \[\set{\seq{x_0,x_1} \in X\times X : x_0 \neq x_1} = {\textstyle\bigcup_{\seq{i_0,i_1} \in A} U_{i_0} \times U_{i_1}}.\]
  Note that we necessarily have \(U_{i_0} \cap U_{i_1} = \varnothing\) for all \(\seq{i_0,i_1} \in A.\)
  Given distinct \(x_0,x_1 \in X,\) we can search through \(A\) for a pair \(\seq{i_0,i_1}\) such that \(\seq{x_0,x_1} \in U_{i_0}\times U_{i_1}.\) 
  Since \(U_{i_0}\times U_{i_1}\) does not intersect the diagonal, we must have \(U_{i_0} \cap U_{i_1} = \varnothing.\)
  So if we define \(\seq{h_0(x_0,x_1),h_1(x_0,x_1)}\) to be the outcome of this search when \(x_0 \neq x_1\) (and arbitrarily when \(x_0 = x_1\)) then \(h_0,h_1:X \times X \to I\) witness that \((X,\mathcal{U},k)\) is effectively Hausdorff.

  Conversely, suppose that \(h_0,h_1: X\times X \to I\) witness that \((X,\mathcal{U},k)\) is effectively Hausdorff.
  Then \[\set{\seq{x_0,x_1} \in X \times X : x_0 \neq x_1} = {\textstyle\bigcup_{\seq{i_0,i_1} \in A} U_{i_0}\times U_{i_1}},\] where \(A\) is the \eset{} \[A = \set{\seq{h_0(x_0,x_1),h_1(x_0,x_1)} : \seq{x_0,x_1} \in X \times X \land x_0 \neq x_1}.\qedhere\]
\end{proof}

\begin{theorem}[\RCA]\label{T:Hausdorff}
  The following are equivalent.
  \begin{enumerate}[\upshape(1)]
  \item Arithmetic Comprehension \textup(\ACA\textup).
  \item Every Hausdorff weak csc space is effectively Hausdorff.
  \item Every Hausdorff strong csc space is effectively Hausdorff.
  \item Every discrete strong csc space is effectively Hausdorff.
  \end{enumerate}
\end{theorem}

\begin{proof}
  The implications (1)\THEN(2)\THEN(3)\THEN(4) are clear. 

  To see that (4)\THEN(1), given an injection \(f:\N\to\N\) we construct a discrete strong csc space \(\csc{X}\) such that if \(\csc{X}\) is effectively Hausdorff then the range of \(f\) is a set.

  For convenience, assume that \(f(0) = 0\) and \(f(1)=1.\) 
  Let \(X = \set{x \in \N : x > 0}.\)
  Given \(\seq{a,b} \in \N^2\) define \[U_{\seq{a,b}} = \set{x \in X : f(a) < f(x) < f(b)}.\]
  Consider the strong csc space \(\csc{X} = (X,\seq{U_{\seq{a,b}}}_{\seq{a,b} \in \N^2},k)\) where \[k(x,\seq{a_0,b_0},\seq{a_1,b_1}) = \begin{cases}
      \seq{a_0,b_0} & \text{if $f(a_0) \geq f(a_1)$ and $f(b_0) \leq f(b_1)$,} \\
      \seq{a_0,b_1} & \text{if $f(a_0) \geq f(a_1)$ and $f(b_0) > f(b_1)$,} \\
      \seq{a_1,b_0} & \text{if $f(a_0) < f(a_1)$ and $f(b_0) \leq f(b_1)$,} \\
      \seq{a_1,b_1} & \text{if $f(a_0) < f(a_1)$ and $f(b_0) > f(b_1)$.} \\
    \end{cases}\]
  This is a discrete space for if \(x \in X\) and \[\begin{aligned}
      f(a) &= \max\set{z \in \ran f : z < f(x)}, &
      f(b) &= \min\set{z \in \ran f : z > f(x)},
    \end{aligned}\] then \(U_{\seq{a,b}} = \set{x}.\)
  
  Suppose that \(h_0,h_1\) witness that \(\csc{X}\) is effectively Hausdorff.
  To determine whether \(z > 0\) is in the range of \(f,\) try to find a pair \(\seq{x_0,x_1} \in X^2\) such that \(f(x_0) \leq z < f(x_1),\) and \(h_0(x_0,x_1) = \seq{a,x_1},\) \(h_1(x_0,x_1) = \seq{x_0,b}\) for some \(a,b \in \N.\)
  Since \(U_{\seq{a,x_1}} \cap U_{\seq{x_0,b}} = \varnothing,\) there is no \(x\) such that \(f(x_0) < f(x) < f(x_1).\) 
  Thus, given such \(\seq{x_0,x_1},\) \(z \in \ran f\) if and only if \(z = f(x_0).\)
  Such a pair \(\seq{x_0,x_1}\) must exist since there is a unique pair \(\seq{x_0,x_1}\) with \[\begin{aligned}
      f(x_0) &= \max\set{y \in \ran f : y \leq z}, &
      f(x_1) &= \min\set{y \in \ran f : y > z},
    \end{aligned}\] is as required.
\end{proof}

\end{document}
