\documentclass[csc]{subfiles}
\setcounter{section}{1}
\begin{document}

\section{Background}

A standard reference for subsystems of second-order arithmetic and their use in reverse mathematics is Simpson~\cite{Simpson09}.
Formal definitions and properties of the basic subsystems \RCA, \WKL, and \ACA\ can be found there.
Due to the highly combinatorial nature of our subject matter, we will make use of several conventions throughout this paper.
The goal of these conventions is to minimize the amount of arithmetical coding in the statement and proofs of our results.

Finite sequences are coded using natural numbers in such a way that all the basic operations on finite sequences are primitive recursive.
There are many ways to accomplish this; see Smory{\'n}ski~\cite{Smorynski77} for a detailed account.
Since the details of this coding are immaterial, we will not impose one particular choice on the reader.

We will use the notation \(X^{<\infty}\) to denote the set of all sequences of elements of the set \(X.\)
The length of a sequence \(x\) will be denoted \(|x|.\)
For a natural number \(n,\) we will write \(X^n\) for the subset of \(X^{<\infty}\) consisting of sequences of length \(n.\)
We will write \(\seq{x_0,\dots,x_{\ell-1}}\) for the sequence of length \(\ell\) whose \(i\)-th element is \(x_{i-1}.\)

\paragraph{\Fset{s}.}
\Fset{s} are coded with finite sequences that enumerate them in increasing order.
We will write \(X^{[<\infty]}\) for the set of all \fset{s} of elements of the set \(X.\)
We will write \(\set{x_0,\dots,x_{\ell-1}}\) for the set whose elements are enumerated by the sequence \(\seq{x_0,\dots,x_{\ell-1}}\) (not necessarily in increasing order).

We will handle \fset{s} as if they were plain sets.
For example, we will write \(x \in a\) to abbreviate \((\exists i < |a|)(x = a_i)\) and we will write \(a \subseteq b\) to abbreviate \((\forall i < |a|)(\exists j < |b|)(a_i = b_j).\)
We will freely take unions and intersections of \fset{s} since these are primitive recursive operations.

\paragraph{\Eset{s}.}
An \emph{\eset{}} is a nondecreasing sequence \(A = \seq{a_n}_{n=0}^\infty\) of \fset{s}.
This sequence is intended to represent the union \(\bigcup_{n=0}^\infty a_n,\) which does not provably exist in \RCA.

We will handle \eset{s} as if they were plain sets.
For example, we will write \(x \in A\) to abbreviate \((\exists n)(x \in a_n)\) and we will write \(A \subseteq B\) to abbreviate \((\forall x)(x \in A \lthen x \in B).\) 
We will also write \(A = B\) to abbreviate \((\forall x)(x \in A \liff x \in B)\) since the pointwise equality of the sequences \(A\) and \(B\) will never be relevant.
We will freely take countable unions and finite intersections of \eset{s} since these can be described in a canonical way.

Note that for every \(\Sigma^0_1\) formula \(\phi(x)\) there is an \eset{} \(A\) such that \((\forall x)(x \in A \liff \phi(x));\) we will abbreviate this by writing \(A = \set{x : \phi(x)}.\)
In fact, we will often use set comprehension to define \eset{s}.
In other words, we will write \(A = \set{x : \phi(x)}\) to abbreviate the routine process by which \(\phi(x)\) is used to construct a nondecreasing sequence \(A = \seq{a_n}_{n=0}^\infty\) of \fset{s} such that \((\exists n)(x \in a_n) \liff \phi(x).\)

\end{document}