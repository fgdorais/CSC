\documentclass[csc]{subfiles}
\setcounter{section}{10}
\begin{document}

\section{Subbases}

There is yet another common approach to prove the Tychonoff Theorem, which goes through the Alexander Subbase Theorem~\cite{Alexander39}.
This approach turns out to be less efficient than Loeb's approach, but the analyis of the method is interesting.

\begin{definition}[\RCA]
  A unformly enumerable sequence \(\seq{W_i}_{i \in I}\) of sub\sset{s} of \(X\) is \emph{compact} if for every \eset{} \(A \subseteq I\) such that \(X = \bigcup_{i \in A} W_i\) there is a \fset{} \(a \subseteq A\) such that \(X = \bigcup_{i \in a} W_i.\)
\end{definition}

\noindent
Thus a weak csc space \(\csc{X}=(X,\mathcal{U},k)\) is compact if and only if its basis \(\mathcal{U}\) is compact as a sequence of sub\sset{s} of \(X.\)

The Alexander Subbase Theorem says that every space with a compact subbase is compact; this fact is provable in \ACA.

\begin{theorem}[\RCA]\label{T:AlexanderACA}
  The following are equivalent:
  \begin{enumerate}[\upshape(1)]
  \item Arithmetic comprehension \textup(\ACA\textup).
  \item Every csc space generated by a compact subbase is basically compact.
  \item Every effectively discrete csc space generated by a compact subbase is finite.
  \end{enumerate}
\end{theorem}

\begin{proof}
  To see that (1)\THEN(2), suppose \(\mathcal{B} = \seq{B_i}_{i \in I}\) is a uniformly enumerable sequence of sub\sset{s} of \(X\) and let \((X,\mathcal{B}^*)\) be the csc space with subbase \(\mathcal{B}.\)
  Suppose further that \(A \subseteq I^{[<\infty]}\) is an \eset{} such that \(X \neq \bigcup_{s \in a} B^*_s\) for every \fset{} \(a \subseteq A.\)
  We will show that \(X \neq \bigcup_{s \in A} B^*_s.\)

  Note that not containing a \fset{} which is a cover is an arithmetical property of finite character.
  By Tukey's Lemma, which was analyzed by Dzhafarov and Mummert~\cite{DzhafarovMummertXX}, we can assume that for every \(t \in I^{[<\infty]} \rem A\) there is a \fset{} \(a \subseteq A\) such that \(X = B^*_t \cup \bigcup_{s \in a} B^*_s.\)
  
  Let \(A_1 = \set{s \in A : |s| = 1}\) and note that \(X \neq \bigcup_{s \in A_1} B^*_s\) by the compactness of \(\mathcal{B}.\)
  Pick \(x \in X \rem \bigcup_{s \in A_1} B^*_s;\) we claim that \(x \notin \bigcup_{s \in A} B^*_s.\)
  
  Suppose instead that there is a \(t \in A\) with \(x \in B^*_t.\)
  Write \(t = \set{t_0,\dots,t_{\ell-1}}\) so that \(B^*_t = B_{t_0} \cap\cdots\cap B_{t_{\ell-1}}.\)
  Observe that \(\set{t_k} \notin A\) for each \(k < \ell.\)
  By maximality of \(A,\) for each \(k < \ell\) there is a \fset{} \(a_k \subseteq A\) such that \(X = B_{t_k} \cup \bigcup_{s \in a_k} B^*_s.\)
  Then \(a = a_0 \cup\cdots\cup a_{\ell-1}\) is a \fset{} contained in \(A\) such that \(X = B^*_t \cup \bigcup_{s \in a} B^*_s,\) which is impossible.

  That (2)\THEN(3) is a direct consequence of Proposition~\ref{P:Discrete}.

  To see that (3)\THEN(1), let \(\seq{f_n}_{n=0}^\infty\) be a sequence of functions \(f_n:\N\to\set{0,1}\) such that \(\lim_{x\to\infty} f_n(x)\) exists for every \(n.\)
  We will show that \(\set{n : \lim_{x\to\infty} f_n(x) = 1}\) is a set.
  Since every \(\Delta^0_2\)-definable \sset{} is of this form, this is enough to establish arithmetic comprehension.

  Define \(\mathcal{B} = \seq{B_s}_{s \in \set{0,1}^{<\infty}}\) by \[B_s = \set{x \in \N : x = |s| \lor (\exists n < |s|)(f_n(x) \neq s_n)}.\]
  Note that \((\N,\mathcal{B}^*)\) is effectively discrete since \(B^*_{\set{0,1}^n} = \set{n}\) for each \(n.\)
  
  By hypothesis~(3), \(\mathcal{B}\) is not compact; say \(A \subseteq \set{0,1}^{<\infty}\) is an \eset{} such that \(\N = \bigcup_{s \in A} B_s\) but \(\N \neq \bigcup_{s \in a} B_s\) for every \fset{} \(a \subseteq A.\)
  Note that if \(s \in A\) then we must have that \(\lim_{x\to\infty} f_n(x) = s_n\) for every \(n < |s|,\) otherwise there would be a \(s \in A\) such that \(B_s\) is cofinite.

  Since \(A\) is infinite, it must contain sequences of arbitrarily long length, thus \[\lim_{x\to\infty} f_n(x) = d \liff (\exists s \in A)(n < |s| \land s_n = d),\] which shows that \(\set{n : \lim_{x\to\infty} f_n(x) = 1}\) is a set by \(\Delta^0_1\)-comprehension.
\end{proof}

\begin{theorem}[\RCA]\label{T:AlexanderWKL}
  The following are equivalent:
  \begin{enumerate}[\upshape(1)]
  \item Weak K{\"o}nig Lemma \textup(\WKL\textup).
  \item Every csc space generated by a compact subbase with a finite cover relation is basically compact.
  \item Every effectively discrete csc space generated by a compact subbase with a finite cover relation is finite.
  \end{enumerate}
\end{theorem}

\begin{proof}
  To see that (1)\THEN(2), suppose \(\mathcal{B} = \seq{B_i}_{i \in I}\) is a compact uniformly enumerable sequence of sub\sset{s} of \(X\) and suppose \(C \subseteq I^{[<\infty]}\) is a finite cover relation for \(\mathcal{B}.\)
  We know from Lemma~\ref{L:SubFiniteCover} that the weak csc space \((X,\mathcal{B}^*)\) has a finite cover relation, so it suffices to show that \((X,\mathcal{B}^*)\) is compact.

  Let \(\seq{s_n}_{n=0}^\infty\) be a sequence of elements of \(I^{[<\infty]}\) such that \(X = \bigcup_{n=0}^\infty B^*_{s_n}.\)
  Let \(T\) be the bounded tree of all sequences \(\seq{i_0,\dots,i_{\ell-1}}\) such that \(i_n \in s_n\) for each \(n < \ell\) and such that \(\set{i_0,\dots,i_{\ell-1}} \notin C.\)

  Note that \(T\) does not have any infinite branches.
  Indeed, if \(\seq{i_n}_{n=0}^\infty\) were to enumerate an infinite branch through \(T,\) then we would have \(\bigcup_{n=0}^\infty B_{i_n} = X\) since \(B^*_{s_n} \subseteq B_{i_n}\) for every \(n,\) but \(\bigcup_{n=0}^{\ell-1} B_{i_n} \neq X\) for every \(\ell\) since \(\set{i_0,\dots,i_{\ell-1}} \notin C.\)
 
  It follows from the Weak K{\"o}nig Lemma that \(T\) must be finite. 
  Let \(\ell\) be the height of \(T,\) then
  \[\bigcup_{n=0}^{\ell-1} B^*_{s_n} = \bigcup_{n=0}^{\ell-1} \bigcap_{i \in s_n} B_i = \bigcap_{\seq{i_0,\dots,i_{\ell-1}} \in s_0\times\cdots\times s_{\ell-1}} \bigcup_{n=0}^{\ell-1} B_{i_n} = X,\] with the convention that empty intersections equal \(X.\)

  That (2)\THEN(3) follows directly from Proposition~\ref{P:EffDiscreteCompact}.

  For (3)\THEN(1), we prove the contrapositive of the Weak K{\"o}nig Lemma.
  Suppose \(T\) is a subtree of \(\set{0,1}^{<\infty}\) which has no infinite branches and let \(X\) be the set of all dead-ends of \(T.\)
  Let \(\mathcal{B} = \seq{B_t}_{t \in T}\) be defined by \[B_t = \set{x \in X : t \nsubseteq x}.\]
  Since \(B_t \cup B_u = X\) exactly when \(t, u\) are incomparable nodes of \(T,\) we see that this is a compact sequence with a finite cover relation.
  The strong csc space \((X,\mathcal{B}^*)\) is effectively discrete since \(\set{x} = \bigcap_{t \in b_x} B_t\) where \(b_x = \set{(x \res i)\cat(1-x(i)) : i < |x|} \cap T.\)
  By hypothesis \((3),\) it follows that \(X\) is finite, hence so is \(T.\)
\end{proof}

\end{document}
